\documentclass[book]{jlreq}

\usepackage[colorlinks=true]{hyperref}

\usepackage{amsmath}

\title{実用プログラミング言語 Labda}
\author{Hexirp}
\date{\today}

\newcommand{\p}[1]{ \mathord{ \left( #1 \right) } }
\renewcommand{\b}[1]{ \mathord{ \left[ #1 \right] } }
\renewcommand{\c}[1]{ \mathord{ \left\{ #1 \right\} } }
\renewcommand{\a}[1]{ \mathord{ \left\langle #1 \right\rangle } }

\begin{document}

\maketitle

\chapter*{序文}

ここは序文である。ここは序文である。ここは序文である。ここは序文である。ここは序文である。ここは序文である。ここは序文である。ここは序文である。

\chapter{型なしラムダ計算}

\section{関数}

関数とは何なのだろうか? 2023 年現在の中学校学習指導要領は、まず最初に「 \( y \) は \( x \) の関数である」とは「 \( x \) の値が定まれば \( y \) の値が定まる」であると定義している。

たとえば、 \( 2 x \) は \( x \) の関数である。 \( x \) の値が \( 4 \) だと定まれば \( 2 x \) の値も \( 8 \) だと定まるからである。もう一つの例として、普通の感覚に反しているかもしれないが、 \( 100 \) は \( x \) の関数である。 \( x \) の値が \( 3 \) だと定まれば \( 100 \) の値は \( 100 \) だと定まるからである。

それまで当然のものとして見なしてきたものを、新しく研究の対象とすることで、数学を先に進めることが出来る。関数も同様である。 \( y \) が \( x \) の関数である時に、その \( x \) と \( y \) の関係そのものを \( f \) と表すことにする。 \( x \) の値が \( a \) である時の \( y \) の値を \( f \p{ a } \) と表すことにする。 \( y = f \p{ x } \) が成り立つ。

ある関数 \( f \) を定義したい時は、「関数 \( f \) は、任意の \( x \) に対して \( f \p{ x } = x + 2 \) が成り立つものとする」と書けばよい。紛れがない時は、「任意の \( x \) に対して」を省略して、単に \( f \p{ x } = x + 2 \) とだけ書いてもよいものとする。

この記法には、いくつかの利点がある。一つ目に、 \( x \) の値が \( a \) である時の \( y \) の値を、 \( f \p{ a } \) として数式の中で簡単に書くことができるようになることである。二つ目に、二つの関数を \( y = x + 1 \) と \( w = z^2 \) のように定義していたのを、 \( f \p{ x } = x + 1 \) と \( g \p{ x } = x^2 \) のように書くことが出来て、変数を減らすことが出来る。三つ目に、 \( f \p{ x } = \p{ x + 1 }^2 \) から \( g \p{ x } = f \p{ x + 1 } + 1 \) を作ることが出来て、古い関数を使った新しい関数を容易に定義することが出来る。四つ目に、 \( f \p{ x } = \p{ x + 1 }^2 \) から \( g \p{ x } = f \p{ f \p{ x } } \) を作ることが出来て、式の長さを短縮することが出来る。

\end{document}
